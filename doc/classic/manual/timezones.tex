\section{Times: EPICS, Local, Greenwich, Daylight Saving} \label{sec:GMT}
The EPICS base software that is used by the IOCs and also the archiver
deals with time as seconds and nanoseconds since January 1, 1990.  This
``\INDEX{EPICS Time}'' is using \INDEX{Greenwich Mean Time} (GMT),
also known as Universal Time Coordinates (\INDEX{UTC}).  So the EPICS
Time stamp 0 stands for 01/01/1990, 00:00:00 UTC.

People living in Germany are typically in a time zone one hour east of
UTC. For them, the EPICS Time stamp 0 translates into January 1, 1990,
at 01:00:00 in the morning. This is one example of ``\INDEX{Local
Time}''. Anybody living in the United States is of course familiar
with time zone conversions ever since you tried to match what's in the
TV Guide with what's actually on TV.

The EPICS base software includes routines for converting EPICS Time
into Local Time and vice versa. Before EPICS R3.14, these routines
used an environment variable \INDEX{EPICS\_TS\_MIN\_WEST} which needed
to be set to the minutes west of UTC. ``-60'' for Germany in the above
example. Since R3.14, this environment variable is no longer
used. The time stamp conversion code in EPICS base now relies on the
operating system and the C/C++ runtime library to handle any time zone
issues.

It is important to remember that the data served by CA Servers is in
EPICS time, that is based on UTC and \emph{not} your local time.  The
ArchiveEngine stores that data as received, which is again in EPICS
time based on UTC.  When the network data server is asked for samples,
those are also based UTC, albeit with a slight shift from a 1990 \INDEX{epoch}
to 1970, simply because this is more convenient to use in most
programming languages. C, C++, Java and perl all include routines for
converting 1970-epoch seconds to local time and back.

Time stamps are only converted to local time when they are displayed
or entered.  The ArchiveExport program will provide you with e.g.\ a
spreadsheet that has a ``Time'' column in local time. The Java Archive
Client will plot the data with a time axis in local time. Whenever you
specify start and end times for a data request, this is done in local
time.

An example of possible consequences: Assume you live in San Francisco,
California (UTC-8), and you receive a CD-ROM with archived data from the SNS
in Oak Ridge, Tennessee (UTC-5). If you want to investigate what happened
at noon, 12:00:00, on 01/01/2004 at the SNS, you will have to query for
09:00:00 to adjust for the different time zones.

The EPICS base code also relies on the operating system services for
\INDEX{daylight saving time} (\INDEX{DST}). At least under RedHat Linux 9 this
seems to work fine when you are inside the United States.
Example: In the US, daylight saving time went into effect on
04/04/2004, 02:00:00, and I archived a "stringin" record which had
device support that converted the time stamp of the record into a string,
including daylight savings information.
The result looks like this when the data is exported again
on the next day, that is at a time where DST is in effect:

\medskip
\begin{tabular}{lll}
EPICS Seconds & Time     & Value \\
\hline
449917196     & 01:59:56 & 04/04/2004 01:59:56.805984000 \\
449917197     & 01:59:57 & 04/04/2004 01:59:57.816036000 \\
449917198     & 01:59:58 & 04/04/2004 01:59:58.826102000 \\
449917199     & 01:59:59 & 04/04/2004 01:59:59.836110000 \\
449917200     & 03:00:00 & 04/04/2004 03:00:00.846121000 (DST) \\
449917201     & 03:00:01 & 04/04/2004 03:00:01.856196000 (DST) \\
449917202     & 03:00:02 & 04/04/2004 03:00:02.867379000 (DST) \\
449917203     & 03:00:03 & 04/04/2004 03:00:03.876315000 (DST) \\
449917204     & 03:00:04 & 04/04/2004 03:00:04.886356000 (DST) 
\end{tabular}

\medskip
\noindent The seconds of the raw EPICS time stamp simply continue to count up
through the transition because UTC is not affected by DST.
(This test was run in the US Mountain time zone, UTC-7, and the
nanoseconds of the EPICS time stamp are omitted.)
The Value of the string record, which contains the local time as the
IOC saw it, jumps from what would have been 02:00:00 to 03:00:00 DST.
When the time stamp is printed (at a later time when DST applies),
the ``Time'' column matches the times in the value string.

After changing the computer's clock to times in January 2004 and 2005,
that is to times outside of DST before and after the data in the
archive, the result is the exact same:
The EPICS time stamp routines use the DST settings that apply for a
given time stamp, not for the current time.

Countries differ in their algorithms for switching to DST, and if your
operating system does not follow the rules in your location, you might
see sudden offsets in time between the wall clock and the archived
data.
