\chapter{Setup, Installation} \label{sec:install}
In general, the archiver toolset should build on Unix-type operating
systems that are supported by EPICS base 3.14.
Specific instructions for Linux (RedHat and Mandrake) as well as Apple
Mac OS~X follow.

\section{Compilation}
The archiver tools use the EPICS build system as for example described
in the ``EPICS: Input/Output Controller Application Developer's
Guide'' for Release 3.14.4.  This means you need the following
prerequisites:
\begin{enumerate}
\item EPICS Base R3.14.4 (or later) needs to be built and installed.\\
      Unless you are running Linux, this might require getting a
      compiler, perl and gnumake.
\item An EPICS extensions setup: ``configure'' directory with the RELEASE
      file appropriately configured to point to your EPICS base installation.
\item ChannelArchiver sources, placed in the ``src'' subdirectory of
      your EPICS extensions directory tree. 
\end{enumerate}

\noindent All the above is either pretty obvious to those who know it
already or beyond this manual to explain, in which case we have to
refer you to the EPICS web site
\href{http://www.aps.anl.gov/epics}{http://www.aps.anl.gov/epics}. 

\medskip

\emph{You need to read and maybe adjust Tools/ToolsConfig.h and
LibIO/ArchiverConfig.h to suit your needs. The most important
parameter in there is ``CONVERSION\_REQUIRED''. Assert that it is
correctly configured! If you use the wrong setting for
CONVERSION\_REQUIRED, you might not notice any problems for some time,
but your data files will be invalid when transferred to another
operating system; your network data server will only provide garbage
data to network clients. 
}

One good thing to do for a sanity check is to run the ArchiveExport
tool on the data provided in the DemoData subdirectory. Try to
retrieve the PV ``DoublePV''. You should see values in the range of
0...10 with time stamps in recent years.  You will not see the exact
time stamps shown in the example, unless you reside in the
``Eastern'' US time zone (UTC-5), as explained in section
\ref{sec:GMT}:

\begin{lstlisting}[keywordstyle=\sffamily]
ArchiveExport DemoData/index DoublePV -text
# Generated by ArchiveExport 2.1.4
# Method: Raw Data

# Time                          DoublePV [a.u.]
03/05/2004 18:54:41.742248000   2.7
03/05/2004 18:57:50.543731200   3
03/05/2004 18:57:50.563760000   3.5
03/05/2004 18:57:50.583788800   3.9
...
\end{lstlisting}

\noindent You will of course only be able to test ArchiveExport after you
have successfully built the archiver toolset, so read on.
In addition to the configuration of the archiver sources
themselves, some open source tools and libraries are required which
are listed in the following subsections. They are included in the
``ThirdParty'' subdirectory of the archiver sources.

With all the required ``ThirdParty'' components in place, building the
ChannelArchiver should be reduced to typing ``make'' in the
ChannelArchiver source directory, followed by the optional setup of the
Matlab/octave glue code which is described in a README file in the
Matlab subdirectory of the ChannelArchiver sources.

\subsection{XML-RPC}
The archiver's network data server uses XML-RPC. The \INDEX{XML-RPC
Setup} requires the installation of at least the C/C++ support. The
Java archive data client includes the JAR files for XML-RPC access
from Java. If you want to access the archive data server from e.g.\ perl,
this would mean you have to install XML-RPC support for perl, too
(one of which is included in the ThirdParty subdirectory of the
archiver sources).

For C and C++, we use xmlrpc-c from
\href{http://xmlrpc-c.sourceforge.net}{http://xmlrpc-c.sourceforge.net}.
The Makefiles in ChannelArchiver/XMLRPCServer assume this to be
installed in the default location, that is under /usr/local.

With RedHat~6.2, xmlrpc-c compiled out of the box.  With everything else,
it has been a varying pain in the eyebrow. Under RedHat~9.0
and Mandrake~10, it ran into a compile-time error that could be fixed
by un-commenting ``using namespace std;'' in the header file which
reported the error.  Under Fedora Core~2 and Mandrake~10, there were
additional errors that can be fixed by replacing includes for
``strstream.h'' with ``strstream'' in the affected files.  Both RedHat
AS or ES and Mandrake 10 needed additional packages, see below.

\NOTE The default serialization code in xmlrpc-c-0.9.9 will serialize
sufficiently small numbers as zero, see details in section \ref{sec:xml:tiny}.
The ``ThirdParty'' subdirectory contains the sources for xmlrpc-c-0.9.9
together with a patch files that corrects the ``using namespace std;'' and
the serialization issue as well as the ``strstream'' problem.
Under RedHat 9 respectively Mandrake 10, the complete installation would
then look as follows:

\begin{lstlisting}[keywordstyle=\sffamily]
cd ChannelArchiver/ThirdParty
tar vzxf xmlrpc-c-0.9.9.tar.gz
cd xmlrpc-c-0.9.9
./configure
# For 64 bit linux, try
# configure --host=i386-linux-gnu
# Patch for serialization and name spaces
patch -p1 <../patch_xmlrpc-c-0.9.9
# Patch for strstream
# SKIP THIS ON REDHAT 9!
patch -p1 <../patch_xmlrpc-c-0.9.9_strstream
make
su
make install
\end{lstlisting}

\noindent Since the first patch also affects Makefiles which are
created as a result of ``configure'', you might prefer to read that
patch file and apply the changes manually when you're not on RedHat 9
or Mandrake 10.

For RedHat Enterprise Linux Workstation 4 (gcc 3.4.4),
I needed to delete an empty ``default:'' tag in a case statement
in line 103 of src/validatee.c, because the compiler considered
it an error.

For Mac OS X, a version with patched configure scripts
\cite{darwinports} that also includes the ``small numbers'' patch
mentioned above is included in a different tar file:
\begin{lstlisting}[keywordstyle=\sffamily]
 tar vzxf xmlrpc-c-0.9.10_darwin.tgz
 cd xmlrpc-c-0.9.10
 ./configure 
 make
 sudo make install
\end{lstlisting}


The XML-RPC library depends on other packages. \emph{The ``configure''
step will report errors in case those are missing.} For RedHat, those
packages are usually included in the distribution but might not have
been installed by default, so look for the RPMs on your RedHat CDs.
The following are also provided in the ThirdParty subdirectory:

\subsubsection{w3c-libwww}
This is needed to compile the XML-RPC library.
For Readhat or Fedora, you can instead install the libwww
and libwww-devel RPMs that come with the OS.
For Mandrake~10, you can use the w3c-libwww sources as is:
\begin{lstlisting}[keywordstyle=\sffamily]
cd ChannelArchiver/ThirdParty
tar vzxf w3c-libwww-5.4.0.tgz
cd w3c-libwww-5.4.0
./configure
make
su
make install
\end{lstlisting}

\noindent Mac OS X requires a patch to the configure script \cite{darwinports}:
\begin{lstlisting}[keywordstyle=\sffamily]
cd w3c-libwww-5.4.0
patch <../w3c-libwww-5.4.0_osx_patch
./configure --enable-shared --enable-static \
    --with-zlib
make
\end{lstlisting}

\subsection{Xerces XML Library}
The \INDEX{Xerces} library is used to parse the XML configuration files of the
ArchiveEngine, IndexTool, and the network data server.
See ``Xerces C++'' under
\begin{center}
\href{http://xml.apache.org/index.html}{http://xml.apache.org/index.html}
\end{center}
or try this direct link:
\begin{center}
\href{http://xml.apache.org/xerces-c/index.html}
     {http://xml.apache.org/xerces-c/index.html}
\end{center}
to get the sources. The Makefiles assume this to be installed under /usr/local.
Example installation under RedHat:
\begin{lstlisting}[keywordstyle=\sffamily]
tar vzxf xerces-c-current.tar.gz 
cd xerces-c-src2_4_0
export XERCESCROOT=`pwd`
cd $XERCESCROOT/src/xercesc
autoconf
./runConfigure -plinux -cgcc -xg++\
               -minmem -nsocket \
               -tnative -rpthread \
                -P/usr/local
make
su
make install
\end{lstlisting}

\noindent For Mac OS X the runConfigure looks like this:
\begin{lstlisting}[keywordstyle=\sffamily]
./runConfigure -p macosx -n native -P /usr/local
\end{lstlisting}

\noindent Newer compilers (like gcc 3.4.4 for RedHat WS 4,
 or gcc 4.0 as used on Mac OS 10.4)
need this patch:
\begin{lstlisting}[keywordstyle=\sffamily]
Add
   #include <xercesc/framework/MemoryManager.hpp>
to
   src/xercesc/util/RefArrayOf.hpp
\end{lstlisting}

% Building Samples
%cd $XERCESCROOT/samples
%./runConfigure -plinux -cgcc -xg++
%make
%ls ../bin
%make clean

\subsection{Expat}
As an inferior alternative to Xerces, the \INDEX{Expat} library is supported
after changing Tools/FUX.h. Expat comes with e.g.\ RedHat~9,
otherwise see
\begin{center}
\href{http://expat.sourceforge.net}{http://expat.sourceforge.net}.
\end{center}
Expat might be a little faster and easier to install, but it does not
offer validation, so it will be up to you to assert that all XML
configuration files are 100\% perfect.

\subsection{XML-Simple}
This XML library for perl is used by the ArchiveDaemon.
It is available from 
\begin{center}
\href{http://www.cpan.org}{http://www.cpan.org}.
\end{center}
\begin{lstlisting}[keywordstyle=\sffamily]
tar vzxf XML-Simple-2.09.tar.gz
cd XML-Simple-2.09
perl Makefile.PL
su
make install
\end{lstlisting}

\subsection{Frontier}
Frontier is an XML-RPC library for perl.
It is used for tests of the XML-RPC Archive Data Server,
including the ArchiveDataClient.pl test script.
Under RedHat 9.0, Fedora 2 and Mandrake 10, it was sufficient to install
Frontier-RPC-0.07b4 like this:
\begin{lstlisting}[keywordstyle=\sffamily]
tar vzxf Frontier-RPC-0.07b4.tar.gz
cd Frontier-RPC-0.07b4
perl Makefile.PL
sudo make install
\end{lstlisting}

\noindent RedHat 6.2 was hopeless because many of the required perl
packages were missing.

Users who want to experiment with the perl client, but lack the
privilege to install in /usr/local can install a private copy like
this:
\begin{lstlisting}[keywordstyle=\sffamily]
...
perl Makefile.PL PREFIX=~
make install
export PERL5LIB=~/lib/perl5/site_perl/5.8.0
\end{lstlisting}
\noindent You might have to adjust the PERL5LIB settings to reflect
your perl version.

\section{Installation}
There are no specific installation procedures for the ArchiveEngine,
ArchiveExport, and most other Channel Archiver components. The
binaries for them end up in the standard EPICS extension directories,
which should therefore be included in the search path. If the archiver
libraries were build as shared libraries, most Unix systems will
require the extensions' lib directory be added to the
LD\_LIBRARY\_PATH. The same applies to other helper libraries like
Xerces that might be in the form of shared libraries.

The usage of the ArchiveEngine and other archiver tools migh require
configuration files, the format of which is described as part of the
tool's dedicated section in this manual.

The ArchiveDataServer requires integration with your web server. The
process is exemplified in the Data Server chapter starting on page
\pageref{sec:dataserver}.

\subsection{DTD Files} \label{sec:dtdfiles}
Many of the configuration files use XML, and document type
definitions are provided in the form of DTD files
(See ArchiveDataServer configuration in \ref{lst:serverconfigdtd},
 ArchiveEngine config.\ in \ref{lst:engineconfigdtd}, ArchiveDaemon in
 \ref{lst:daemonconfigdtd}, ArchiveIndexTool config.\ in
 \ref{lst:indexconfigdtd}).
You are \emph{strongly} encouraged to reference these DTD files in all
your XML files, and to use the validating Xerces XML library, so that
all your XML get valiated while the ChannelArchiver tools use them.
This means that your XML files need to include a DOCTYPE declaration
that points to the location of the respective DTD file.
In practice, there are at least three ways to accomplish this:
\begin{enumerate}
\item Whereever you create an XML file, you copy the DTD into the same
  directory. Then you can refer to the DTD like this: 
  \begin{lstlisting}[keywordstyle=\sffamily]
  <!DOCTYPE engineconfig SYSTEM "engineconfig.dtd">
  \end{lstlisting}
  \noindent Not the best idea because you need multiple copies of the
  DTD and this is hard to maintain in case the DTD gets updated.
\item You install the DTD files in a common location in the local file
  system, e.g.\ in ``/arch''. Then you can refer to the DTD like this: 
  \begin{lstlisting}[keywordstyle=\sffamily]
  <!DOCTYPE engineconfig
  SYSTEM "/archr/engineconfig.dtd">
  \end{lstlisting}
  \noindent This setup is in use at the SNS. If you use the archive tools
  on more than one computer, each machine might require a copy of the DTDs.
\item You install the DTD files in the directory tree of a web server
  that is accessible to all your computers.
  Then you can refer to the DTD via a URL like this: 
  \begin{lstlisting}[keywordstyle=\sffamily]
  <!DOCTYPE engineconfig
  SYSTEM "http://webserver/archdtd/engineconfig.dtd">
  \end{lstlisting}
  This centralizes the installation, but you now have the added dependency on
  the web server.
\end{enumerate}
