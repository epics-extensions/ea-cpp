\section{\INDEX{Sample Mechanisms}} \label{sec:sampling} % -----------------------------
This section describes the basic sampling mechanisms and their issues,
while the actual configuration of an archive engine is detailed in section
\ref{sec:engineconfig}.

Each ChannelAccess Server provides time-stamped data. An IOC for
example stamps each value when the corresponding record is
processed.  These time-stamps offer nano-second granularity. Most
applications will not require the full accuracy, but some
hardware-triggered acquisition, utilizing interrupts on a fast CPU,
might in fact put the full time stamp resolution to good use.

Other examples where exact time stamps are of interest are channels
for user setpoints or faults. These channels might not change for a
long time, but when they change, it is often important to know the
exact time of the setpoint adjustment or fault.

The archive engine tries to preserve the time stamps provided by
the CA server, which results in some difficulties for the various
sampling modes as well as the user of the resulting data.

\NOTE The values dumped into the data storage will not offer much
indication of the sampling method. In the end, we only see values with
time stamps. If for example the time stamps of the stored values
change every 20 seconds, this could be the result of a monitored
channel that happened to change every 20 seconds. We could also face a
channel that changed at 10~Hz but was only sampled every 20 seconds. 

\subsection{\INDEX{Monitored Sampling}}
In this mode, the ArchiveEngine requests a CA monitor, i.e.\ it
subscribes to changes of the channel, then stores all the values that the server
sends out. The CA server configuration determines when values are sent.

\medskip

\begin{figure}[htb]
\begin{center}
\InsertImage{width=1.0\textwidth}{images/times}
\end{center}
\caption{\label{fig:times}Time Stamps and Sampling}
\end{figure}

\noindent This sounds simple enough; The archive captures each change of the channel,
with the original time stamp.
In fig.~\ref{fig:times}, black diamonds represent the actual changes of the
channel that would be archived with monitored sampling.
In practice, there are several problems:
\begin{itemize}
\item Channels like setpoints or fault indicators rarely change in an
      operational machine.
      But if a channel never changes, you get no new data.
      Since the Engine also logs whenever ChannelAccess declares the
      channel 'disconnected', we can assume that the last value in the archive
      is still valid, until eventually a new value arrives.
      In reality, however, users often suspect that something is wrong if they
      see no new sample in the archive for days.
      And often, they are correct.
\item If the channel changes very rapidly, you get a lot of data, filling the
      disk with probably more detail than ever needed.
      In any case, the engine cannot simply
      allocate more and more memory to buffer incoming samples between
      writing them to disk, because a rogue channel that suddenly sends
      a burst of values would then cause the engine to use up all the computer's
      memory. Consequently, one needs to configure an expected period
      between value changes, based upon which the engine prepares its
      internal memory buffers. If many more values arrive, some are ignored.
\end{itemize}

\subsection{\INDEX{Scanned Sampling}}
In this mode, the ArchiveEngine periodically requests a value from
the CA server, for example once every minute. The value returned by
the CA server is added to the archive. So even if the value does not
change, we take a sample, which gives more of a warm fuzzy feeling than
the monitored mode.
To prevent saving the exact same value over and over again, wasting disk space,
only 'repeat' counts are stored: Just before a new value is logged, the
previous, unchanged value is logged with special status/severity values
to indicate the \INDEX{repeat count}.
If the value never changes, the engine increments the repeat count
to a configurable maximum, then logs it and starts again.
So even for values that never change, the archive should contain
a periodic sample.

For most slowly varying values like water temperatures or a long-term
archive of operating voltages etc., this is a very practical mode.
Possible issues:
\begin{itemize}
\item
Not suitable for many "fault" indicator type channels:
If a brief fault happens between sample times, the engine never notices
it.
\item
Since each "scan" is handled
by actually requesting a value from the CA server and logging the
result, i.e.\ a round-trip network request for each value,
scanning should be limited to maybe once a minute or slower.
\item
Since each sample is stored with its original time stamp, 
users are often surprised when retrieving archived data.
Assuming we scan a channel every 30~seconds, the following
happens: About every 30 seconds, the ArchiveEngine requests and stores
the current value of the channel \emph{with its original time stamp!}.
So while the ArchiveEngine might take a sample at
\begin{center}
14:53:30, 14:54:00, 14:54:30, 14:55:00, ...,
\end{center}
it stores the time stamps that come with the values, not the nice looking,
equidistant sample times. Depending on network
delays and the scan rate of the record, that value might have been time-stamped
some time before the engines sample time.
In Fig.~\ref{fig:times} those times happened to be
\begin{center}
14:53:29.091,  14:53:59.092, 14:54:29.094,  14:54:59.095, ...
\end{center}
\end{itemize}

\subsection{\INDEX{Scanned using Monitors}}
The ArchiveEngine is configured to scan periodically,
e.g.\ one sample every 5 seconds. But instead of actively requesting a
value from the CA server at this rate, it establishes a monitor and
only saves a value every 5 seconds.

To clarify the "Scan" with and witout monitors, assume our data source changes
at 1~Hz. If we want to store a value every 30 seconds, it is most efficient to
send a 'read'-request every 30 seconds. If, on the other hand, we want
to store a value every 5 seconds, it is usually more effective to
establish a monitor, so we automatically receive updates about every
second, and simply \emph{ignore} 4 of the 5 values.

When configuring a channel, the user only selects ``Scan'' with a sampling rate.
The decision between the simple scan and the scan based on monitors is
automatic, depending on the scan period. If the scan period is smaller,
i.e.\ faster than the \emph{get\_threshold} configuration
parameter described in section \ref{sec:getthreshold}, 
monitors are used.

Possible issues:
\begin{itemize}
\item
For scan rates of maybe 1 to 30 seconds, there is usually a performance
advantage because of the reduced network traffic in comparison to
the plain scan with round-trip network requests.
But in order to give repeat count information similar to the plain scan,
the engine still needs to periodically check the channel for newly arrived
monitors. At fast scan periods like 0.1 seconds, this results in higher
CPU load than basic monitored sampling.
\item
The engine has problems with merging data from the internal scan with
incoming monitors.
Assume that we scan every 10 seconds.
At the end of one such scan interval, we might notice that we received a monitor,
time stamped in the previous sample interval, so it is logged.
At the next sample interval, we might not have received anything new,
so we increase the repeat count.
At an even later sample interval, we might still not have received anything
new, and having accumulated the maximum repeat count, we log a special 'repeat'
sample, time-stamped at the sample interval time.
Just a split second later, we might receive a new monitor, stamped before the
last sample interval, but because of network delays it only arrived now.
Since the archive already contains the special 'repeat' value, this new sample
goes back-in-time and cannot be logged as is.
To get it into the archive, its time stamp is adjusted to match the
preceding 'repeat' value. 
\end{itemize}

\section{Sensible Sampling}


The data source configuration and sampling need to be coordinated.  In
fact the whole system needs to be understood. When we deal with water
tank temperatures as one example, we have to understand that the
temperature is unlikely to change rapidly. Let us assume that it only
varies within 30...60 seconds. The analog input record that reads the
temperature could be configured to scan every 2 seconds. Not because
we expect the temperature to change that quickly but mostly to provide
the operator with a warm and fuzzy feeling that we are still reading
the temperature: The operator display will show minuscule variations
in temperature every 2 seconds.  An ArchiveEngine that is meant to
capture the long-term trend of the tank temperature could then sample
the value every 60 seconds.

On the other extreme could be channels for vacuum readings along linac
cavities. The records that read them might be configured to scan as
fast as the sensing devices permit, maybe beyond 10~Hz, so that
interlocks on the IOC run as fast as possible. Their dead bands (ADEL
and MDEL) on the other hand are configured to limit the data rate that
is sent to monitoring CA clients: Only meaningful vacuum changes are
sent out, significantly reducing the amount of data sent onto the
network.  The ArchiveEngine can then be configured to monitor the
channel: During normal operation, when the vacuum is fairly stable, it
will only receive a few values, but whenever the vacuum changes
because of a leak, it will receive a detailed picture of the event.

Another example is a short-term archive that is meant to store
beam position monitor (BPM) readings for every beam pulse. The records
on the IOC can then be configured with ADEL=-1 and the ArchiveEngine
to use monitors, resulting in a value being sent onto the network and
stored in the archive even if the values did not change. The point
here is to store the time stamps and beam positions for each beam
pulse for later correlation. Needless to say that this can result in a
lot of data if the engine is kept running unattended. The preferred
mode of operation would be to run the engine only for the duration
of a short experiment.

\NOTE The scanning of the data source and the ArchiveEngine run in
parallel, they are not synchronized.
Example: If you have a record scanned every second and want to capture
every change in value, configuring the ArchiveEngine to scan every
second is {\bfseries not} advisable:
Though both the record and the ArchiveEngine would scan every
second, the two scans are not synchronized and rather unpredictable
things can happen. Instead, the "Monitor" option for the ArchiveEngine
should be used for this case.
